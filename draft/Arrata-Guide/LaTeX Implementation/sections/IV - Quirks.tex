\documentclass[../main.tex]{subfiles}

\graphicspath{{\subfix{../images/}}}

\begin{document}

    Quirks are the backbone of any character. They help you as a player or GM define who exactly a character is, how they operate, and how you should be representing them. The point of Quirks is to allow a degree of freedom in roleplaying a character without letting you lose what makes that character unique as a person.

    Quirks are usually a single word or a very short phrase that defines a particular characteristic of someone. They are not intended to be stereotypes or absolute rules of how a character works. Instead, they define the boundaries, biases, beliefs, and any aspect of the character that's relevant to the story.
      
    \section{Quirk Types}

        Quirks are divided into three categories under the three classical rhetorical appeals: \textbf{Ethos}, \textbf{Pathos}, and \textbf{Logos}. Each category defines a set of Quirks and what they usually do for a character. By building a character with at least one or two Quirks in each category, you're almost guaranteed to have at least a half-interesting person.

        \begin{itemize}
            \item [\textbf{Ethos}] Ethos expresses a character's Ethical, Moral, Societal, and Religious beliefs and context. Often they contain information about their past and how they're currently seen by the society they live in today. Ethos Quirks are usually what gets a person into trouble; what they use to stir the pot and cause conflict.
            
            \item [\textbf{Pathos}] Pathos deals with a character's emotional situation - how they act around other people and with what level of apathy or empathy they approach different tasks. Pathos Quirks tend to define things that may seem simple or stereotypical but can be used in much more nuanced ways when combined with other Quirks.
            
            \item [\textbf{Logos}] Logos is how a character makes decisions; it's their inner voice that drives their actions step by step through whatever mess the other Quirks put them into.
        \end{itemize}

    \subsection{Boons and Flaws}

        Quirks can offer \emph{Boons} and \emph{Flaws} which allow for relevant rolls to be modified. When a check is made such that a Quirk's Boons seem relevant, then that roll will gain a level of advantage. In opposition, if a Quirk's Flaws seem more relevant, they will gain a level of disadvantage. Note that this is not exclusive; Quirks that have a Boon in a scenario could also have a relevant Flaw, and therefore the roll would have a level of advantage and disadvantage.

        \emph{Note: Bring your GM treats; they may consider your PC's Boons.}
        \begin{mdframed}[style=Arrata]
            Here's an example of the \emph{Veritable} Quirk:

            {\em
                \textbf{Name}: Veritable
            
                \textbf{Defines}:
            
                \begin{itemize}
                    \item Being a genuine article or item.
                    \item A real instance of something believed gone or impossible.
                    \item Acting genuinely or truthfully.
                \end{itemize}
            
                \textbf{Boons}: People will often trust or believe you. They might see you as an ally when things are wrong in the world.
            
                \textbf{Flaws}: You may often disclose things you should not. When there is great abundance, you may be seen as archaic.
            }
        \end{mdframed}

    \section{Fighting Quirks}

        Doing something that isn't what a character would normally do is incredibly interesting, but only if such an event is justifiable, otherwise Quirks would have no meaning other than to provide you with advantage and disadvantage. In the event where a character might reasonably consider and even act in a way contrary to a Quirk, we say they're \emph{Fighting the Quirk}.

        To clarify, Fighting the Quirk is an event where a character might say,
        
        \begin{center}
            ``\emph{Do I want to be me? Do I accept who I am, or should I change?}''
        \end{center}
    
        Fighting against Quirks is the key to \emph{Change} in Arrata. You as a player are the controller of your character and are ultimately the one who pilots the fate of your character. Part of that fate is deciding if a character \emph{Accepts} or \emph{Rejects} their Quirks.

    \subsection{Acceptance and Rejection}

        \emph{Acceptance} and \emph{Rejection} are measures of how much a character likes or dislikes a particular Quirk. Utilizing a Quirk in ways that demonstrate not only a reliance on the Quirk, but trust and belief in that aspect of the character is likely to increase your character's Acceptance of that Quirk. Doing the opposite; Fighting a Quirk (and succeeding), increases its Rejection.
    
        Acceptance and Rejection function like stats, although they don't have a Quality, and aren't rolled. Instead, they act in opposition to each other; for every level of Acceptance, acts that fight against the Quirk gain a level of disadvantage. For every level of Rejection, acts that utilize the Quirk's Boons gain a level of Disadvantage.

        But, they also offer relief when used to further themselves; acts that are faithful to a Quirk and use its Boons gain levels of advantage equal to the Quirk's Acceptance, and acts that fight the Quirk gain equal levels of advantage to its Rejection.


    \section{Intuition}\label{intuition}

        Intuition is a point system that seeks to reward good character crafting and storytelling; when Quirks are roleplayed well, and when the conflict in the story is dealt with in interesting ways.

        A single ``Intuition point'' is awarded to a PC by the GM when the player of that character does one of the following:

    \begin{itemize}
        \item Roleplays a Quirk well,
        \item Roleplays a scenario well,
        \item Creates a particularly funny or interesting scenario,
        \item Fights against a Quirk successfully.
    \end{itemize}

        It's also important to note that for the given methods of gaining Intuition, if a particular Quirk is a reason why Intuition is being gained, then the Intuition will go into that Quirk's Intuition category. If there isn't a Quirk that caused it, the player may choose which category they want the Intuition to go to freely.

        How often and in what volume Intuition is given out is dependent on the GM, but every player should be earning 1-2 Intuition points per 3-4 hours of play.

    \section{Argos}

        \begin{mdframed}[style=Arrata]
            {\em
                Argos is a city in the Greek Peloponnese, the same island Sparta is situated on. Argo, the ship Jason used with his Argonauts, was the vessel by which he carried out his journey. Many terrible things happen on this adventure to find the Golden Fleece, but ultimately, they retrieve the Fleece and return to Greece where Jason assumes his father's throne. This story is short-lived though, and the people reject Jason and his wife, driving them into solitude. Jason breaks his vows to his wife in exile, and she takes his new wife's life as well as their child's ascending to Mount Olympus. Jason returns to his land, where the Argo is on display. As he rests next to it, a part of it breaks loose, crushing and killing Jason.
            }
        \end{mdframed}
        
        \textbf{Argos} in Arrata is the drive a character has. Their goal, mission, and ultimate weakness. It is their source of power and the destination where they're ``retired''; their final resting place. It's from their Argos that a story is driven.

        Argos is often a sentence or short phrase, written from the perspective of the character who has that Argos. The written Argos should be short, astute, and clear in its goal; it should be a stopping point the character deeply desires and wishes for. You should write a character's Argos as if it were their final words, and in a way only they could fully understand.

        This is because an Argos is incredibly dangerous for a character. It can drive them into stupidly dangerous situations, taking on foes far more powerful than themselves. It can also be corrupted and turned against them, used to manipulate a character for the benefit of another.

        Argos should incorporate a character's Quirks; if they're \emph{Caliban}, then so should their Argos, if they're \emph{Corrupt}, then their Argos should be underhanded, if they're \emph{Cursed}, then they should be fighting with or against that curse.

        Argos provides your characters with a special power most others don't have: \emph{Purpose}. With this purpose, their actions will become more likely to succeed, and actions that directly serve towards moving that character closer to their Argos should be offered a level of advantage by the GM. This is the power that Argos provides, but actions that go against Argos are severely punished. If a character moves against their Argos, ignoring the goal they've sought, they will be considered ``Failing'', and it's the job of everyone at the table to notify their player that they're not playing their character to their Argos.

        If still, they continue to ignore or fight against their Argos, it's the GM's job to intervene. Characters who have a purpose but don't care for it will lose it, and revert to their original, purposeless existence. Remove these characters (and probably their player) from the table immediately. This is one rule that is immutable in Arrata: \textbf{Evict purposeless characters and the players who lead them to this}. If the situations and events that have occurred make it such that they lose their Argos, that is acceptable. But if a player's actions drag their character time and again away from their purpose, then they're actively breaking the rules of this system. Take necessary action and remove them.

        Not all Argos are noble. People fight and die over incredibly stupid reasons constantly. A hope with Arrata is to allow you to see those purposes and their reasons, even if you might disagree with them, and understand them better.

    \subsection{Completing Argos}

        If the Argos of a character is made true, then we say that the character has \emph{Completed their Argos}. Their journey in this story is over, the control of the player over the character is no longer needed for them to live in this world. Discuss with your group then; what should we do with this character? Do they retire, move out somewhere nice and settle down? Do we send them off on another, tangential adventure? Will the characters remaining hear of them again? Wonder what happy, or bittersweet ending would be appropriate for them. Ultimately, this is a decision between a player and the GM, with a heavy lean towards the player's opinion. As a player, you're bargaining for the best end for this person you've made come true. As a GM, you're trying to retire and fit that character into the background as well as possible.

        Some characters might choose to stay and see the others' journeys through to completion. They might turn and abandon them in a bleak moment of vengeance. As the character's player, you should choose an option best fitting the character and see them through to the end well: it's your last responsibility to that character.

    \subsection{Breaking Argos}

        In opposition, there might come a time when an Argos is made void; a family seeking to be reunited is destroyed, a wish for peace is turned into an eternal war, and vengeance is proved wrong in its assumptions. When this happens, we say that the character's \emph{Argos is Broken}. Often, this is a traumatic event for them, where emotions take precedence, and stupid decisions are made in desperation.

        When Argos is broken, a decision is to be made immediately: what does the character do? Do they sink into endless despair, do they go on a rampage, do they turn and silently leave? Do any of the other characters stop them? A broken Argos can be mended, and a new one can be found. This is a critical part of \emph{change}, but before that can be done, the trial of overcoming such a deep loss should be difficult. The GM should assign a few checks to see in what ways the character degenerates and what they lose. If it seems to be a total loss, \emph{toss the character aside}. They're gone, and there's nothing you can do about it.

\end{document}