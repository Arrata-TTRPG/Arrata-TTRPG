\documentclass[../main.tex]{subfiles}

\graphicspath{{\subfix{../images/}}}

\begin{document}

    Quirks are the backbone of any character. They help you as a player or GM define who exactly a character is, how they operate, and how you should be representing them. The point of Quirks is to allow a degree of freedom in roleplaying a character without letting you lose what makes that character unique as a person.

    Quirks are usually a single word or a very short phrase that defines a particular characteristic of someone. They are not intended to be stereotypes or absolute rules of how a character works. Instead, they define the boundaries, biases, beliefs, and any aspect of the character that's relevant to the story.
      
    \section{Quirk Types}

        Quirks are divided into three categories under the three classical rhetorical appeals: \textbf{Ethos}, \textbf{Pathos}, and \textbf{Logos}. Each category defines a set of Quirks and what they usually do for a character. By building a character with at least one or two Quirks in each category, you're almost guaranteed to have at least a half-interesting person.

        \begin{itemize}
            \item [\textbf{Ethos}] Ethos expresses a character's Ethical, Moral, Societal, and Religious beliefs and context. Often they contain information about their past and how they're currently seen by the society they live in today. Ethos Quirks are usually what gets a person into trouble; what they use to stir the pot and cause conflict.
            
            \item [\textbf{Pathos}] Pathos deals with a character's emotional situation - how they act around other people and with what level of apathy or empathy they approach different tasks. Pathos Quirks tend to define things that may seem simple or stereotypical but can be used in much more nuanced ways when combined with other Quirks.
            
            \item [\textbf{Logos}] Logos is how a character makes decisions; it's their inner voice that drives their actions step by step through whatever mess the other Quirks put them into.
        \end{itemize}

    \subsection{Boons and Flaws}

        Quirks can offer \emph{Boons} and \emph{Flaws} which allow for relevant rolls to be modified. When a check is made such that a Quirk's Boons seem relevant, then that roll will gain a level of advantage. In opposition, if a Quirk's Flaws seem more relevant, they will gain a level of disadvantage. Note that this is not exclusive; Quirks that have a Boon in a scenario could also have a relevant Flaw, and therefore the roll would have a level of advantage and disadvantage.

        \emph{Note: Bring your GM treats; they may consider your Boons more often.}
        \begin{mdframed}[style=Arrata]
            {\em
                \lipsum[1]
            }
        \end{mdframed}

    \section{Fighting Quirks}

        Doing something that isn't what a character would normally do is incredibly interesting, but only if such an event is justifiable, otherwise Quirks would have no meaning other than to provide you with advantage and disadvantage. In the event where a character might reasonably consider and even act in a way contrary to a Quirk, we say they're \emph{Fighting the Quirk}.

        To clarify, Fighting the Quirk is an event where a character might say,
        
        \begin{center}
            ``\emph{Do I want to be me? Do I accept who I am, or should I change?}''
        \end{center}
    
        Fighting against Quirks is the key to \emph{Change} in Arrata. You as a player are the controller of your character and are ultimately the one who pilots the fate of your character. Part of that fate is deciding if a character \emph{Accepts} or \emph{Rejects} their Quirks.

    \subsection{Acceptance and Rejection}

        \emph{Acceptance} and \emph{Rejection} are measures of how much a character likes or dislikes a particular Quirk. Utilizing a Quirk in ways that demonstrate not only a reliance on the Quirk, but trust and belief in that aspect of the character is likely to increase your character's Acceptance of that Quirk. Doing the opposite; Fighting a Quirk (and succeeding), increases its Rejection.
    
        Acceptance and Rejection function like stats, although they don't have a Quality, and aren't rolled. Instead, they act in opposition to each other; for every level of Acceptance, acts that fight against the Quirk gain a level of disadvantage. For every level of Rejection, acts that utilize the Quirk's Boons gain a level of Disadvantage.

        But, they also offer relief when used to further themselves; acts that are faithful to a Quirk and use its Boons gain levels of advantage equal to the Quirk's Acceptance, and acts that fight the Quirk gain equal levels of advantage to its Rejection.


    \section{Intuition}\label{intuition}

        Intuition is a point system that seeks to reward good character crafting and storytelling; when Quirks are roleplayed well, and when the conflict in the story is dealt with in interesting ways.

        A single ``Intuition point'' is awarded to a PC by the GM when the player of that character does one of the following:

    \begin{itemize}
        \item Roleplays a Quirk well,
        \item Roleplays a scenario well,
        \item Creates a particularly funny or interesting scenario,
        \item Fights against a Quirk successfully.
    \end{itemize}

        It's also important to note that for the given methods of gaining Intuition, if a particular Quirk is a reason why Intuition is being gained, then the Intuition will go into that Quirk's Intuition category. If there isn't a Quirk that caused it, the player may choose which category they want the Intuition to go to freely.

        How often and in what volume Intuition is given out is dependent on the GM, but every player should be earning 1-2 Intuition points per 3-4 hours of play.

    \section{Argos}



    \subsection{Completing Argos}

    \subsection{Breaking Argos}

    \section{Roleplaying and Quirks}

\end{document}