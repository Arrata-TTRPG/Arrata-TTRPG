\documentclass[../main.tex]{subfiles}

\graphicspath{{\subfix{../images/}}}

\begin{document}

    Quirks are the backbone of any character. They help you as a player or GM define who exactly a character is, how they operate, and how you should be representing them. The point of Quirks is to allow a degree of freedom in roleplaying a character without letting you lose what makes that character unique as a person.

    Quirks are usually a single word or a very short phrase that defines a particular characteristic of someone. They are not to be taken as stereotypes or absolute rules for that aspect of that character, but as a frame of reference from which you can jump off to roleplay that character. 
      
    \section{Quirk Types}

    Quirks are divided into three categories under the three classical rhetorical appeals: \textbf{Ethos}, \textbf{Pathos}, and \textbf{Logos}. Each category defines a set of Quirks and what they usually do for a character. By building a character with at least one or two Quirks in each category, you're almost guaranteed to have at least a half-interesting person.

    \subsubsection{Ethos}

    Ethos expresses a character's Ethical, Moral, Societal, and Religious beliefs and context. Often they contain information about their past and how they're currently seen by the society they live in today. Ethos Quirks are usually what gets a person into trouble; what they use to stir the pot and cause conflict.

    \subsubsection{Pathos}

    Pathos deals with a character's emotional situation - how they act around other people and with what level of apathy or empathy they approach different tasks. Pathos Quirks tend to define things that may seem simple or stereotypical but can be used in much more nuanced ways when combined with other Quirks.

    \subsubsection{Logos}

    Logos is how a character makes decisions; it's their inner voice that drives their actions step by step through whatever mess the other Quirks put them into.

    \section{Intuition}

    Intuition is a point system that rewards good roleplaying. Both when Quirks are roleplayed well, and when the conflict in the story is dealt with in interesting ways.

    How often and in what volume Intuition is given out is dependent on the GM, but every player should be earning 1-2 Intuition points per 3-4 hours of play.

    It's also important to note that for the given methods of gaining Intuition, if a particular Quirk is a reason why Intuition is being gained, then the Intuition will go into that Quirk's Intuition category. If there isn't a Quirk that caused it, the player may choose which category they want the Intuition to go to freely.

    Intuition is awarded to a PC by the GM when the player of that character does one of the following:

    \begin{itemize}
        \item Roleplays a Quirk especially well,
        \item Roleplays a scenario especially well,
        \item Creates a particularly funny or interesting scenario,
        \item Fights against a Quirk successfully.
    \end{itemize}

    Note the last situation. Fighting against Quirks; doing something that isn't what a character would normally do is incredibly interesting. That's not just a moment to go ``Woah'', it's a moment that ushers in a question about that character: ``\emph{Do I want to be me?}''.

    \section{Fighting and Accepting Quirks}
    
    Fighting against Quirks is the key to \emph{Change} in Arrata. You as a player are the controller of your character and are ultimately the one who pilots the fate of your character. Part of that fate is deciding if a character \emph{Accepts} or \emph{Rejects} their Quirks. They're measured as point values with a minimum of 0, and both increase or decrease depending on how you roleplay that particular Quirk.

    \subsection{Acceptance}

    Acceptance is how much a character likes this particular Quirk. Utilizing a Quirk in ways that demonstrate not only a reliance on the Quirk, but a trust and belief in that aspect of the character is likely to increase your character's acceptance of that Quirk. 
    
    Acceptance functions like a stat, although it doesn't have a Quality, and it isn't rolled, meaning Acceptance is represented as just a number or Quantity. Acceptance levels just like any stat; checks are gained when your character gains advantage from using the Quirk. 

    \subsection{Boons and Flaws}

    Quirks can offer \emph{Boons} and \emph{Flaws} which allow for relevant rolls to be modified. When a check is made such that a Quirk's Boons could be advantageous, then the 

    
    \subsection{Rejection}



    \section{Argos}

    \subsection{Completing Argos}

    \subsection{Breaking Argos}

    \section{Roleplaying and Quirks}

\end{document}