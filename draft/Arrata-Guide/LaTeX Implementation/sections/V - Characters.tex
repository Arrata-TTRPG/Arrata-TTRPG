\documentclass[../main.tex]{subfiles}

\graphicspath{{\subfix{../images/}}}

\begin{document}

    This chapter deals with what and who characters are; what are they composed of, how are they compiled, and how do we go about creating them? 

    \section{Character Stats}

        As a component of a character, a character's stats define \emph{what} they are in the world; what they're capable of, and what they aren't. As an overview, these are aspects of characters that don't deserve an entire section:

        \begin{itemize}
            \item [\textbf{Name:}] What they're known as. Self-explanatory, but could also include pseudonyms and nicknames.
            \item [\textbf{Player:}] Who's responsible for playing the character.
            \item [\textbf{Age:}] How long has this character lived in this world.
            \item [\textbf{Stock:}] The race or species this character hails from.
        \end{itemize}

    \subsection{Core Stats}

        The core stats of a character are generalized parts of them: how strong, fast, and smart they are. These stats are used for generalistic scenarios that involve less tact and more brute force to solve a problem, and are often used to determine the magnitude of a particular check's effectiveness if it succeeds.

        The core stats are divided into two groups of three, the first being \emph{mental stats}:

    \begin{itemize}
        \item[ \textbf{Will:}] General knowledge or common sense, ability to learn, and ability to resist urges. 
        \item [\textbf{Perception:}] Ability to see, smell, hear, and detect the environment and subtleties of the world.
        \item [\textbf{Conscious:}] Ability to process and understand information as well as the speed of cognition.
    \end{itemize}

        The second group is the {\em physical stats}:

    \begin{itemize}
        \item [\textbf{Power:}] Strength and physical capability.
        \item [\textbf{Speed:}] Agility and swiftness to commit actions.
        \item [\textbf{Forte:}] Physical health, conditioning, and ability to resist malicious infection.
    \end{itemize}

    \subsection{Stat Resources}

        Stat resources (SRs) are a measure of something that a character has or something they're enduring; things like a curse, wealth, or the support of the people, that's been abstracted into a stat. This can be extremely useful to quantify something not necessarily quantifiable and then to allow something quantified to become part of the system in a way that allows it to be used in rolls.
        
        What stat resources a character has and how they function are things to discuss with the GM and other players. If you feel uncomfortable with this mechanic or would rather use exact numbers or roleplay to represent these things, then feel free to ignore stat resources altogether.

        Stat resources are divided into being either \emph{Finite} or \emph{Infinite}.

        \subsubsection{Finite SRs}
        
        Finite stat resources are things like wealth, the favor of the people, rations for a journey, and so on. These are things that are destroyed, lost, or diminished as you employ them. 

        Finite stat resources generally aren't rolled alone. They're used to add onto and modify rolls, but at a cost. When rolling a finite stat resource, \emph{any failures} reduce the Quantity of the resource by 1. Once the finite resource's Quantity hits 0, the stat resource cannot be used until its Quantity is increased to 1 or above.

        \emph{Note: Finite stats can be used to abstract strange currencies.}
        \begin{mdframed}[style=Arrata]
            {\em
                \lipsum[1]
            }
        \end{mdframed}

        \subsubsection{Infinite SRs}

        Infinite stat resources are things like curses, reputation, and unusual abilities. infinite stat resources are more complicated than Finite ones, and aren't used up when you use them; usually, it's much the opposite; infinite stat resources are more like core stats, they typically level up like traditional stats (see: {\nameref{changing stats}}) and are used to solve problems all on their own.

        \emph{Note: Infinite stats can be used to represent non-fatal disease.}
        \begin{mdframed}[style=Arrata]
            {\em
                \lipsum[1]
            }
        \end{mdframed}

    \subsection{Skills}

    \section{Change}

    \subsection{Changing Stats}\label{changing stats}

    Leveling is a mechanical process through which characters improve their abilities by performing actions and learning from their experiences. Most stats in the game are level-able, but it's important to consult your GM to confirm whether things like \emph{Resource stats} are eligible for leveling.

    \subsection{Check `Points'}

    Each time you make a check for a stat that can be leveled, you gain a `check' (point). These points accumulate slowly, increasing by +1 for every check made. Once the check pool reaches a value of 2 times the Quantity of the stat, the stat immediately levels up! You can then rejoice as you increase the Quantity by +1, and reset the check pool back to 0.

    Depending on your character sheet, the check pool may be represented as a designated area to record the current value, such as filling in empty circles or iterating a number field. Regardless of the method, it's crucial to keep track of your checks, as this is the \emph{only} way a stat can be leveled.

    \subsection{Optional: Intuition}
    
    This is optional but recommended:

    When sufficient check points are available for a stat to level, you don't immediately level that stat. Instead, you must spend an intuition point of any category alongside with all of your check points (\emph{even if there's more than needed!}). I offer this as an interesting addition to leveling to try and force more numbers-inclined players to focus on roleplay while still being able to experience that precious dopamine hit. It also forces the players to spend their characters' Intuition more wisely.

    \subsection{Changing Quirks}
\end{document}