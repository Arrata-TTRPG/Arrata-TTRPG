\documentclass[../main.tex]{subfiles}

\graphicspath{{\subfix{../images/}}}

\begin{document}

    \section{Why Dice?}

    Dice are tools that are used to generate random numbers, which are in turn used to determine the outcome of certain scenarios. By adjusting things like how we count the value of each die, how many dice are rolled, and what special rules apply to them, we turn completely random, arbitrary values into probabilities that reflect the upper and lower bounds of a particular thing.

    \section{Dice Notation}

    When using and discussing quantities of dice, often the term Dice Notation may be used. This refers to a system that helps define two things about the dice being rolled:

    \begin{itemize}
        \item How many dice are to be rolled, represented as $Y$.
        \item How many sides the dice being rolled have; represented as $X$.
    \end{itemize}

    This is composed with a $D$ in between, which stands for dice, in the form $YDX$, although I prefer and will use a lowercase $d$ for the rest of this book. Individual dice are often written without the $Y$ value as $dX$. 
    
    \emph{Note: 100-sided dice are often a composition of $d10+d10\times10$.}
    \begin{mdframed}[style=Arrata]
        \begin{align*}
            \text{1 6-sided die: }                  & 1d6 \text{ or } d6    \\
            \text{3 dice with 20 sides each: }      & 3d20                  \\
            \text{14 dice with 6 sides each: }      & 14d6                  \\
            \text{100 dice with 100 sides each: }   & 100d100
        \end{align*}
    \end{mdframed}

    \section{Rolled Dice}

    When a roll is made, the result in this book will be recorded in parentheses () and each die's result will be separated by commas. These values are chosen at random for this book.
    
    \emph{Note: ellipses $(a,\ldots,b)$ are used to represent a large amount of data.}
    \begin{mdframed}[style=Arrata]
        \begin{center}
            $YdX = (r_{0},r_{1},\ldots,r_{Y})$ where $r_{k}$ is the rolled value of the die $X_{k}$
        \end{center}
        \begin{equation*}
            \begin{gathered}
            \text{I rolled a six-sided die and got a 4:}                                    \\
                    1d6  = (4)                                                              \\
            \text{I rolled 3 twenty-sided dice and got 5, 15, and 20:}                      \\
                    3d20 = (5, 14, 20)                                                      \\
            \text{I rolled 100 one-hundred-sided dice and got 99, 65, \ldots, 23, and 55:}\\
                    100d100 = (99, 65, \ldots, 23, 55)                                      
            \end{gathered}
        \end{equation*}
    \end{mdframed}

    \section{Addition and Subtraction}

    There will be cases where a roll would be given or have lost dice to roll, in which case we represent the change to a quantity of dice as $+/-XD$, where $X$ is the number of dice being added or subtracted and $D$ (always capitalized) is denoting that $X$ represents a quantity of dice.

    Separately, if two different-sided quantities of dice are added, there is no attempt to unify them into a single roll. Instead, they are left in their separate states and written as $Y_{1}dX_{1} + Y_{2}dX_{2}$.
    \\
    \begin{mdframed}[style=Arrata]
        \begin{align*}
            \text{I gained $3d6$ for my $6d6$ roll: }   & 6d6 + 3D = 9d6            \\
            \text{I lost $2d20$ for my $4d20$ roll: }   & 4d20 - 2D = 2d20          \\
            \text{I gained $100d6$ for my $5d8$ roll: } & 100d6 + 5d8 = 100d6 + 5d8    
        \end{align*}
    \end{mdframed}

    \section{Exploding Dice}

    There are also cases where dice can ``explode''. This means that when the maximum possible value of a die is rolled, the value of that die is kept, and you can add $+1D$ to the roll, rolling one more die. This can theoretically repeat infinitely, although the probability of that is essentially impossible.

    To denote a roll as exploding, add an exclamation point, $!$, to the front. Here are a few examples, not that they are summed to show how the value of the exploded dice affected the outcome:
    
    \emph{Note: Rolled dice that have a modifier applied to them are bolded (\textbf{6}).}
    \vspace*{-0.325cm}
    \begin{mdframed}[style=Arrata]
        \begin{align*}
            !3d6    & = (\textbf{6} + 2 + 5) =\; (\textbf{6} + 2 + 5) + !1d6 = 13 + (4) = 17  \\
            !2d20   & = (\textbf{20} + \textbf{20}) =\; 40 + !2d20 = 40 + (10 + 15) = 65      \\
            !6d2    & = (1 + \textbf{2} + 1 + 1 + \textbf{2} + \textbf{2}) =\; 9 + !3d2  = \dots    
        \end{align*}
    \end{mdframed}

    \section{Evil Dice}

    In opposition to exploding dice, Arrata will deal with {\em Evil dice}. Evil dice are denoted by adding an upside-down exclamation point, !`. Instead of giving the roll an additional die to roll and add to the sum, Evil dice give an extra $D1$ that subtracts from the roll.

    \emph{Note: Evil dice and Exploding dice can happen simultaneously!}
    \begin{mdframed}[style=Arrata]
        \begin{align*}
            \text{!`}2d20  & = (\textbf{1} + 5) = 6 - \text{!`}1d20 = 6 - (10) = -4                  \\
            \text{!`}6d6   & = (4 + 5 + 3 + \textbf{1} + 2 + 6) = 19 -  \text{!`}1d6 = 19 - (6) = 13 \\
            \text{!`}3d10  & = (\textbf{1} + \textbf{1} + \textbf{1}) = 3 - \text{!`}3d10 = 3 - (\textbf{1} + 2 + \textbf{1}) = \ldots
        \end{align*}
    \end{mdframed}

    \section{Dice Pools}

    Arrata functions on {\em Dice Pools}. This is a way of rolling dice that focuses not on the sum of the values of the dice rolled, but by comparing each value to a constant, $C$.

    \section{Conditionals}

    For Dice Pools, conditionals are used along with a given constant $C$ to achieve a specific effect. For Arrata, this conditional is the $>$ operator. This is used to count the number of dice rolled that are larger than $C$.

    For example:

    \begin{itemize}
        \item $4d20>10 = (12, 13, 4, 1)>10 = 2$
        \item $5d4>1 = (1, 4, 2, 1, 3)>1 = 3$
        \item $2d10>9 = (4, 7)>9 = 0$
        \item \dots
    \end{itemize}

    This counted sum can be used for several schemas, and the value of $C$ can be used to further tune probabilities. Arrata makes heavy use of conditionals for its systems.

    \section{The d6}

    Arrata uses the d6 as its primary die and no others. It's a convenient die as they're extremely stackable, provide a decent window of probabilities, and are often very cheap and numerous, which is excellent for Arrata because Dice Pool-based rolls can call for 10+ dice at once.

    Because we know all rolls in Arrata use the d6, whenever a Quantity of dice is discussed, dice notation will not be used. Instead, the roll will be composed into a \textbf{Stat}.

\end{document}