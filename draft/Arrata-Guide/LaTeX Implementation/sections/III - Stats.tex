\documentclass[../main.tex]{subfiles}

\graphicspath{{\subfix{../images/}}}

\begin{document}

    Now that we've established the basic rules of dice, we can translate those into the mechanics, different parts of Characters, and the components that make them up.

    \section{Definition}

    A stat is a composition of two elements:

    \begin{itemize}
        \item \textbf{Quality}: The $C$ constant used for a conditional roll.
        \item \textbf{Quantity}: The number of d6s to roll.
    \end{itemize}

    Stats are values that represent the capability of a single part of something or someone. They represent, in a statistical sense, the upper and lower bounds of what that part can do.

    \subsection{Quantity}

    Quantity has essentially already been defined; it is the Quantity of dice rolled, specifically in d6s. It specifies the $Y$ component of $YdX$ or the value of the dice pool.

    In a more character-focused sense, Quantity represents the capacity to do what a particular stat does. It defines the upper bound for the stat's capability.

    \subsection{Quality}

    Quality is the $C$ constant used for a conditional roll for the dice pool. In Arrata, Quantity comes in 3 levels:
    
    \begin{itemize}
        \item \textbf{B}asic:  $C$ = 3.
        \item \textbf{A}dept:  $C$ = 2.
        \item \textbf{S}uperb: $C$ = 1.
    \end{itemize}

    For the value of stats, refer to the first level of the name of the Quality. For example:

    \begin{itemize}
        \item $10d6>3$ is $B$ Quality.
        \item $4d6>2$ is $A$ Quality.
        \item $5d6>1$ is $S$ Quality.
    \end{itemize}

    Quality is special in terms of characters' stats as it represents not how much a person could do with a stat, but how easily they reach that maximum.

    \subsection{Composition}

    Stats in Arrata are not written in dice notation, instead, they are composed in the format $BX$ where $B$ is the letter of the Quality and $X$ is the value of the Quantity.

    Here are some examples with the Arrata-composed stat and its equivalent dice notation form:

    \begin{itemize}
        \item $B6 = 6d6>3$
        \item $A100 = 100d6>2$
        \item $S40000 = 40000d6>1$
    \end{itemize}

    \section{Checks}

    \subsection{Obstacle}

    \subsection{Successes and Failures}

\end{document}