\documentclass[../main.tex]{subfiles}

\graphicspath{{\subfix{../images/}}}

\begin{document}

    Now that we've established the basic rules of dice, we can translate those into the mechanics, different parts of Characters, and the components that make them up.

    \section{Definition}

    A stat is a composition of two elements:

    \begin{itemize}
        \item \textbf{Quality}: The $C$ constant used for a conditional roll.
        \item \textbf{Quantity}: The number of d6s to roll.
    \end{itemize}

    Stats are values that represent the capability of a single part of something or someone. They represent, in a statistical sense, the upper and lower bounds of what that part can do.

    \subsection{Quantity}

    Quantity has essentially already been defined; it is the Quantity of dice rolled, specifically in d6s. It specifies the $Y$ component of $YdX$ or the value of the dice pool.

    In a more character-focused sense, Quantity represents the capacity to do what a particular stat does. It defines the upper bound for the stat's capability.
    
    \subsection{Quality}

    Quality is the $C$ constant used for a conditional roll for the dice pool. In Arrata, Quantity comes in 3 levels:
    
    \begin{itemize}
        \item \textbf{B}asic:  $C$ = 3.
        \item \textbf{A}dept:  $C$ = 2.
        \item \textbf{S}uperb: $C$ = 1.
    \end{itemize}

    For the value of stats, refer to the first level of the name of the Quality. For example:

    \begin{itemize}
        \item $10d6>3$ is $B$ Quality.
        \item $4d6>2$ is $A$ Quality.
        \item $5d6>1$ is $S$ Quality.
    \end{itemize}

    Quality is special in terms of characters' stats as it represents not how much a person could do with a stat, but how easily they reach that maximum.

    \subsection{Composition}

    Stats in Arrata are not written in dice notation, instead, they are composed in the format $BX$ where $B$ is the letter of the Quality and $X$ is the value of the Quantity. Here are some examples with the Arrata-composed stat and its equivalent dice notation form:

    \begin{itemize}
        \item $B6 = 6d6>3$
        \item $A100 = 100d6>2$
        \item $S40000 = 40000d6>1$
    \end{itemize}

    Now that stats are defined, we can discuss what exactly they're used for.

    \section{Checks}

    A critical part of roleplaying is meeting something that is a challenge for the character to overcome; something that they may or may not be able to do. When this happens; when an action is contested, a \textbf{Check} is called for. Dice are rolled and compared to a {\em difficulty level} to determine the outcome, which the GM will interpret.

    \subsection{Success and Failure}

    Because Arrata uses dice pools and comparisons, Arrata works on a binary success/failure schema. Quality defines the threshold for what a success is; if a die is rolled and is greater than its Quality constant, then the die rolled is counted as a success. This is done for each die you roll and the number of successes is summed up. Any die whose value rolled is equal to or less than the Quality is called a failure. The sum of the failures of a roll is not usually used for anything and that operation will be stated ahead of time, so when you roll, don't worry about summing them up.

    For example:

    \begin{itemize}
        \item Rolling $B2$: $(4, 2)>3 =$ 1 Success, 1 Failure.
        \item Rolling $A5$: $(2, 6, 1, 3, 5)>2 =$ 3 Successes, 2 Failures.
        \item Rolling $S4$: $(6, 2, 5, 4)>1 =$ 4 Succeses, 0 Failures.
    \end{itemize}

    \subsection{Obstacle}

    In Arrata we refer to the {\em difficulty level} as \textbf{Obstacle}. When making a check, this value will be provided by the GM, by a specific subsystem, or it may not be provided at all. Obstacle defines the lower bound of the number of successes needed to {\em pass} the check. If you roll successes below this value, you will {\em fail} the check.

    For nomenclature's sake, Obstacle is shortened to $Ob\; X$, where $Ob$ stands for Obstacle and $X$ represents the value of the Obstacle for the check. For an entire check, it is written in the form $Stat\mathrm{\; vs \;}Ob\; X$.

    Here are a few examples:

    \begin{itemize}
        \item Rolling $B2\mathrm{\; vs \;}Ob\; 1: (2, 2)>3\mathrm{\; vs \;}1 =$ 0 Successes vs $Ob\; 1$ = $failure$.
        \item Rolling $A4\mathrm{\; vs \;}Ob\; 2: (5, 6, 3, 5)>2\mathrm{\; vs \;}2 =$ 4 Successes vs $Ob\; 2$ = $pass$.
        \item Rolling $S6\mathrm{\; vs \;}Ob\; 4: (1, 5, 1, 2, 3, 4)>1\mathrm{\; vs \;}4 =$ 4 Successes vs $Ob\; 4$ = $pass$.
    \end{itemize}

    \subsection{Intent}

    When a check is called for, {\em Intent} must be defined. State what exactly it is your character intends to do and what they hope will happen by doing it; that will be used to define the {\em difficulty level}. The GM will then determine the outcome:

    \begin{itemize}
        \item If you {\em pass} the check,
        \item If you {\em fail} the check,
        \item If you have extra successes/failures.
    \end{itemize}

    \pagebreak

    \subsection{Outcomes}

    \paragraph{Passing a Check}

    \paragraph{Failing a Check}

    \paragraph{Extra Successes and Failures}
    
    \subsection{Advantage}

    \subsection{Disadvantage}

    \section{Types of Checks}

    \subsection{Close-ended Checks}

    \subsection{VS Checks}

    \subsection{Open-ended Checks}

    \subsection{Help}

    \section{Character Stats}

    \subsection{Core Stats}

    \subsection{Skills}

\end{document}