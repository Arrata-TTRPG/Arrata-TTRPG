\documentclass[../main.tex]{subfiles}

\graphicspath{{\subfix{../images/}}}

\begin{document}

    Now that we've established the basic rules of dice, we can translate those into the mechanics, different parts of Characters, and the components that make them up. A stat is a composition of two elements:

    \begin{itemize}
        \item \textbf{Quality}: The $C$ constant used for a conditional roll.
        \item \textbf{Quantity}: The number of d6s to roll.
    \end{itemize}

    Stats are values that represent the capability of a single part of something or someone. They represent, in a statistical sense, the upper and lower bounds of what that part can do.

    \section{Quantity}

    Quantity has already been defined; it is the number of dice rolled, specifically in d6s. It specifies the $Y$ component of $YdX$ or the value of the dice pool. In a more character-focused sense, Quantity represents the capacity to do what a particular stat does. It defines the upper bound for the stat's capability.

    Quantity is an {\em uncapped} value, meaning that Quantity values can be arbitrarily large, from 1 to whatever lies just below infinity. Luckily, you won't need to purchase $\inf - 1$ d6s, as Arrata will almost always deal with Quantity values from 1 to 10. In rare cases, Quantities might be in excess of 20, but those are extremely rare and represent supernatural forces beyond conventional limits.

    \section{Quality}

    Quality is the $C$ constant used for a conditional roll for the dice pool. In Arrata, Quantity comes in 3 levels:
    
    \begin{itemize}
        \item \textbf{B}asic:  $C$ = 3.
        \item \textbf{A}dept:  $C$ = 2.
        \item \textbf{S}uperb: $C$ = 1.
    \end{itemize}

    For the value of stats, refer to the first level of the name of the Quality. For example:

    \begin{itemize}
        \item $10d6>3$ is $B$ Quality.
        \item $4d6>2$ is $A$ Quality.
        \item $5d6>1$ is $S$ Quality.
        \item \dots
    \end{itemize}

    Quality is special in terms of characters' stats as it represents not how much a person could do with a stat, but how easily they reach that maximum. Most stats will be of Basic Quality, being Adept or Superb means that specific stat is beyond conventional ability; usually representing some sort of prodigal ability or technologically advanced method.

    \section{Composition}

    Stats in Arrata are not written in dice notation, instead, they are composed in the format $BX$ where $B$ is the letter of the Quality and $X$ is the value of the Quantity. Here are some examples with the Arrata-composed stat and its equivalent dice notation form:

    \begin{itemize}
        \item $B6 = 6d6>3$
        \item $A100 = 100d6>2$
        \item $S40000 = 40000d6>1$
        \item \dots
    \end{itemize}

    Now that stats are defined, we can discuss what exactly they're used for.

    \section{Checks}

    A critical part of roleplaying is meeting something that is a challenge for the character to overcome; something that they may or may not be able to do. When this happens; when an action is contested, a \textbf{Check} is called for. Dice are rolled and compared to a {\em difficulty level} to determine the outcome, which the GM will interpret.

    Checks are the core of the system, knowing when a check occurs and what to do are critical pieces of information for GMs and players alike. Not only do they drive forwards the story, but checks are also used to challenge things about characters, which allows them to learn and improve.

    \section{Success and Failure}

    Because Arrata uses dice pools and comparisons, Arrata works on a binary success/failure schema. Quality defines the threshold for what a success is; if a die is rolled and is greater than its Quality constant, then the die rolled is counted as a success. This is done for each die you roll and the number of successes is summed up. Any die whose value rolled is equal to or less than the Quality is called a failure. The sum of the failures of a roll is not usually used for anything and that operation will be stated ahead of time, so when you roll, don't worry about summing them up.

    For example:

    \begin{itemize}
        \item Rolling $B2$: $(4, 2)>3 =$ 1 Success, 1 Failure.
        \item Rolling $A5$: $(2, 6, 1, 3, 5)>2 =$ 3 Successes, 2 Failures.
        \item Rolling $S4$: $(6, 2, 5, 4)>1 =$ 4 Succeses, 0 Failures.
        \item \dots
    \end{itemize}

    \section{Obstacle}

    In Arrata we refer to the {\em difficulty level} as \textbf{Obstacle}. When making a check, this value will be provided by the GM, by a specific subsystem, or it may not be provided at all. Obstacle defines the lower bound of the number of successes needed to {\em pass} the check. If you roll successes below this value, you will {\em fail} the check. If an Obstacle value is higher than your stat's Quantity, you automatically fail the check.

    For nomenclature's sake, Obstacle is shortened to $Ob\; X$, where $Ob$ stands for Obstacle and $X$ represents the value of the Obstacle for the check. For an entire check, it is written in the form $Stat\mathrm{\; vs \;}Ob\; X$.

    Here are a few examples:

    \begin{itemize}
        \item Rolling $B2\mathrm{\; vs \;}Ob\; 1: (2, 2)>3\mathrm{\; vs \;}Ob\; 1 =$ 0 Successes vs $Ob\; 1$ = $failure$.
        \item Rolling $A4\mathrm{\; vs \;}Ob\; 2: (5, 6, 3, 5)>2\mathrm{\; vs \;}Ob\; 2 =$ 4 Successes vs $Ob\; 2$ = $pass$.
        \item Rolling $S6\mathrm{\; vs \;}Ob\; 4: (1, 5, 1, 2, 3, 4)>1\mathrm{\; vs \;}Ob\; 4 =$ 4 Successes vs $Ob\; 4$ = $pass$.
        \item \dots
    \end{itemize}

    \section{Intent}

    When a check is called for, {\em Intent} must be defined. State what exactly it is your character intends to do and what they hope will happen by doing it; that will be used to define the {\em difficulty level}. The GM will then determine the outcome:

    \begin{itemize}
        \item If you {\em pass} the check,
        \item If you {\em fail} the check,
        \item If you have extra successes/failures.
    \end{itemize}

    \section{Extra Successes}

    Some checks may have boons if you have successes extra. for example:

    \begin{displayquote}
        {\em A cook is making a large volume of stew with their $B5$ cooking skill. The GM declares that with their ingredients, the Obstacle of the check will be $Ob\; 2$. The cook rolls and gets all 5 successes! The GM says that because the cook not only met but surpassed the Obstacle, the resultant stew is incredibly delicious, and the patrons consuming it are mesmerized.}
    \end{displayquote}

    There are no extra consequences to having successes under the Obstacle of the check.
    
    \section{Advantage}

    Some situations will happen where a side in a check has an advantage in doing their task. For example; exploiting the environment, having a relevant Quirk, playing into your Argos, having the high ground in a fight, and getting Help from another character would all induce a level of advantage. 

    When advantage is had, the rolling side with advantage turns their roll into an open-ended roll. In addition, if multiple sources provide a level of advantage higher than 1, or the roll was already open-ended, then the extra levels of advantage turn into $+1D$ each.

    With open-ended rolls, remember that any maxes of the die (6) will add $+1D$ to the roll. The 6s that are rolled are also counted as successes. Open-ended rolls in Arrata are denoted with a $!$ in front of the roll.

    For example:

    \begin{itemize}
        \item $!B3 =\; !(6, 4, 6)>3 = 2$\; Successes $+\; !B2 = $\; 2 Successes $ +\; !(4, 2)>3 = $ 3 Successes.
        \item $!G4 =\; !(6, 2, 3, 5)>2 = 3$\; Successes $+\; !G1 = $\; 3 Successes $+\; !(4)>2 = $ 4 Successes.
        \item $!S6 =\; !(6, 4, 6, 3, 5, 3)>1 = 6$\; Successes $+\; !S2 = $\; 6 Successes $+\; !(2, 2)>1 = $ 8 Successes.
        \item \dots
    \end{itemize}

    \section{Disadvantage}

    There are also situations where the inverse may be true; the terrain is unfavorable, your Quirks are opposed to the action, it opposes your Argos, having the low ground in combat, and enemies harrying you would all induce a level of disadvantage.

    Disadvantage imposes Evil dice to the roll, and may also be obtained in levels. Past the first level of disadvantage, or if the roll already has Evil dice, the check will have +1 Ob imposed per level of extra disadvantage.

    Evil dice subtract -1 Success from rolls that result in a minimum value (1). In Arrata, they're denoted with a !` in front of the roll.

    For example:
    % Fix examples.
    \begin{itemize}
        \item $!B3 =\; !(6, 4, 6)>3 = 2$\; Successes $+\; !B2 = $\; 2 Successes $ +\; !(4, 2)>3 = $ 3 Successes.
        \item $!G4 =\; !(6, 2, 3, 5)>2 = 3$\; Successes $+\; !G1 = $\; 3 Successes $+\; !(4)>2 = $ 4 Successes.
        \item $!S6 =\; !(6, 4, 6, 3, 5, 3)>1 = 6$\; Successes $+\; !S2 = $\; 6 Successes $+\; !(2, 2)>1 = $ 8 Successes.
        \item \dots
    \end{itemize}

    \section{Help, All at Once}

    There comes a time when two or more characters will be working towards the same intent at the same time. It could be that some are attempting to help others, which is called Help, or that they're doing a sensitive task in parallel, which is called All at Once. Choose a character to act as the leader of the roll - this person should be the one who is relying the most on the other characters - the weakest link in the scenario.

    Have the non-leading characters roll first, summing the success {\em and} failures. Subtract the successes from the failures, and give that level of advantage to the leader of the roll. If the number is negative, give that level of disadvantage instead. Also, note down a check for all characters rolling here.

    Here is an example of Help:

    \begin{displayquote}
        {\em Agnar woke up at the bottom of a pit with a large boulder on top of him! He's uninjured, but at an awkward angle; luckily, his comrade Steven has arrived to help! Since Agnar is in the disadvantaged position and is the one in need of help, he'll be making this supporting roll with his A5 Power stat, and Steven will be leading the roll with his weaker B4 Power stat. The GM puts that, to free Agnar, the Ob will be 3. Failing to meet that Ob will result in the boulder crushing Agnar's foot, injuring him!

        Agnar rolls first: 3 successes - 2 failures, a net of +1! That means Steven gets to roll with a level of advantage! Steven rolls: !(6, 1, 5, 2), 2 successes but he gets to roll an extra B1 because of the help from Agnar: (4), making 3 successes! They both roll the boulder off Agnar, an act that seems to have won Steven some free liquor tonight!}
    \end{displayquote}

    And one of All at Once:

    \begin{displayquote}
        {\em Steven hears the scraping of boots up ahead. Unfortunately, both Agnar and himself lost their weapons in the fall and will need to deploy stealth if they hope to avoid getting gutted. Unfortunately, this calls for a Stealth check, one that falls under All at Once, and Agnar has the tact and Stealthiness of a pregnant horse (B2). Steven sighs, and prepares his A6 Stealth roll. The GM declares that this Stealth check will be an Ob 4.
        
        He gets 5 successes - 1 failure! A net +4 advantage for Agnar! That means Agnar rolls a !B5: (6, 6, 3, 4, 1). 3 Successes so far, but he gets 2 more from the open-endedness of the roll: (4, 1). That makes 4 successes! Through some miracle, Steven manages to compensate for Agnar's bumbling mess of a stealth attempt, and they sneak past whatever's prowling these halls in one- well, two pieces.}
    \end{displayquote}

    \section{Character Stats}

    \subsection{Core Stats}

    \subsection{Skills}

\end{document}