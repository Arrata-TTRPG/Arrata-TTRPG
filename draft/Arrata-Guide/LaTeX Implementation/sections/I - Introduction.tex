\documentclass[../main.tex]{subfiles}

\graphicspath{{\subfix{../images/}}}

\begin{document}

    This part contains all of the core mechanics of Arrata; detailing Roleplaying, Characters, their components, dice rolling, and how Characters change.

    \section{What is Arrata?}

    \subsection{Entomologically}
    The word Arrata is a misspelling of errata, the plural of erratum; a list of mistakes in a written document. The word was chosen as it embodies the spirit best of Arrata: \textbf{Change Through Purpose.}

    Failure, mistakes, blunders, they're all a critical part of change and finding purpose. Without the monolithic power that is a purpose, achieving the deeds you may soon find yourself dreaming up is nearly impossible. I hope Arrata will be able to allow you to experience and explore these ideas, and I hope you will be able to change too.
    
    \subsection{Literally}
    Arrata is a Tabletop-Roleplaying Game (TTRPG) system with a heavy focus on allowing players freedom in how they roleplay while also giving them a helping hand in figuring out who exactly their character is, and why they choose to struggle with the horrific realities of their world.

    Arrata is {\em not} a numbers heavy game. There is dice rolling and the occasional division, but almost all of the math is incredibly straight forward unless you want it to become more complicated. To that end, this first part will focus on the underlying systems and ideas that define Arrata, with the idea that you can construct the game you want to play with further subsystems that fit the setting and intent of a given world.
    

    \section{What is Roleplaying?}

    Roleplaying is the act of putting yourself in the shoes of someone else. It's a form of improv where your job is to emulate and represent a character. There are many parts to roleplaying, such as the accent the character has, the ways they interact with different people, how they solve problems, etc.
    
    Anyone roleplaying a character should know as much about that character as they can if they hope to provide a reliable and accurate representation of them, and Arrata provides many tools that allow you to do this, along with helping you develop that character as they experience new things.

    \section{Game Masters (GMs)}

    Game Masters (GMs) are a critical part of any roleplaying system. Their job is to:

    \begin{itemize}
        \item Understand the rules as thoroughly as possible.
        \item Roleplay Non-Player Characters (NPCs).
        \item Be courteous and fair to their Players.
        \item Provide a story and setting.
        \item Describe:
        \begin{itemize}
            \item The outcomes of rolls,
            \item The environment,
            \item Reactions and consequences.
        \end{itemize}
    \end{itemize}
    
    The GM's job is to be the world-engine, describing and defining what the world is, how it looks, and how it interacts with the players and their actions.
    
    As a GM, you have the most responsibility as you must know and/or improv everything the players and their characters will come to discover in your world. You also have the most amount of power, it's important to keep things fair and reasonable. It's also your job to act responsibly as a mediator if any disagreement or argument happens related to the game. The GM's rulings are absolute, so be sure to write them down if you know you'll come across them again.
    
    GMs typically orchestrate Campaigns and Sessions, which is an added layer of responsibility. To compensate this, they are afforded the right to remove Players they think are a poor fit or are being a nusance to the group.

    \section{Players and their Characters (PCs)}

    Players are the people in charge of Player Characters (PCs); their job is to be the ``soul'' driving their character in the direction most appropriate for them. Players are charged with the following responsibilities:

    \begin{itemize}
        \item Roleplaying their character.
        \item Being courteous to the Game Master and fellow players.
        \item Knowing the rules within reason.
        \item Following the rules and decisions of the Game Master.
        \item Being honest about rolls and their character sheet.
    \end{itemize}

    \subsection{Player Characters (PCs)}

    PCs are the protagonists of Arrata. Their existence is to provide everyone with a different point of view on the collective story being told, they are a vessel for your ideas and dreams, and so you must treat them with care and respect. 

    \subsection{Bad Players}

    Being a bad player is easier than one might assume. Here are a few examples and ideas of how you might mitigate their appearance and effect on the game:
    
    \begin{itemize}
        \item When it comes to roleplaying, it's very easy to become emotionally invested in your character and the characters around you, but that can be dangerous and detrimental to everyone else. Becoming too invested into a character can lead to an over-reliance on tropes and a refusal to acknowledge their need to change. It can lead to outbursts of anger and potential sabotage to see a specific vision come true. In order to avoid this, create characters that are dissimilar to yourself and that you might not necessarily agree with.
        
        \item People will often have dissimilar visions for characters, and some players may end up trying to define another PC. This is unacceptable behavior, both in that it violates duty that is charged of the Player, and that is violates their freedom to play a character as best as they can. Your job is to maintain and play your own character, if you feel someone else is roleplaying theirs poorly, or that they're attempting to hijack your character, speak with your GM and re-establish your authority as the player of your PC.
        
    \end{itemize}

\end{document}