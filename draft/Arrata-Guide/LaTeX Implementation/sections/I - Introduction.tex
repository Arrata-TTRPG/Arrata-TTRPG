\documentclass[../main.tex]{subfiles}

\graphicspath{{\subfix{../images/}}}

\begin{document}

    This part contains all of the core mechanics of Arrata; detailing Roleplaying, Characters, their components, dice rolling, and how Characters change.

    \section{What is Arrata?}

    \subsection{Etymologically}
    The word Arrata is a misspelling of errata, the plural of erratum; a list of mistakes in a written document. The word was chosen as it embodies the spirit of Arrata: \textbf{Change Through Purpose.}

    Failure, mistakes, and blunders; they're all critical parts of change and finding purpose. Without the monolithic power that is a purpose, achieving the deeds you may soon find yourself dreaming up is nearly impossible. I hope Arrata will be able to allow you to experience and explore these ideas, and I hope you will be able to change too.
    
    \subsection{Literally}
    Arrata is a Tabletop-Roleplaying Game (TTRPG) system with a heavy focus on allowing Players freedom in how they roleplay while also giving them a helping hand in figuring out who exactly their character is, and why they choose to struggle with the horrific realities of their world. It is designed to be played with 3 - 6 Players and a single GM, with a bare minimum of 1 Player and 1 GM.

    Arrata is {\em not} a numbers-heavy game. There is dice rolling and the occasional division, but almost all of the math is incredibly straightforward unless you want it to become more complicated. To that end, this first part will focus on the underlying systems and ideas that define Arrata, with the idea that you can construct the game you want to play with further subsystems that fit the setting and intent of a given world.
    

    \section{What is Roleplaying?}

    Roleplaying is the act of putting yourself in the shoes of someone else. It's a form of improv where your job is to emulate and represent a character. There are many parts to roleplaying, such as the accent the character has, the ways they interact with different people, how they solve problems, etc. There is a lot to learn about roleplaying, but the best way to do so is to get into a game with a character and practice by doing it.

    \section{Game Masters}

    Game Masters (GMs) are a critical part of any roleplaying system. Their job is to:

    \begin{itemize}
        \item Understand the rules as thoroughly as possible.
        \item Roleplay Non-Player Characters (NPCs).
        \item Be courteous and fair to their Players.
        \item Provide a story and setting.
        \item Describe:
        \begin{itemize}
            \item The outcomes of rolls.
            \item The environment.
            \item NPCs and their actions.
            \item Reactions and consequences.
        \end{itemize}
    \end{itemize}
    
    The GM is the world engine, describing and defining what the world is: how it looks, smells, tastes, and sounds, and how it interacts with the Players' Characters and their actions.
    
    As a GM, you have the most responsibility; orchestrating sessions and campaigns, managing NPCs, handling disputes, etc. Your Players are counting on you to prepare and improvise as well as you can and if you can't do those things, I suggest being a Player.
    
    \subsection{GM Authority}
    Game Masters are to be afforded extra rights over the Players. They will have to make rulings and decisions for the Players, and should act as a mediator; thus these rulings are to be respected and treated as the new rule of law unless otherwise changed by the GM.

    However, it is important not to overstep your authority as the GM. Punishing Players unfairly or making nonsensical rulings are unacceptable. If you find yourself under a GM making such decisions, the best course of action is typically to discuss the issue with the other Players and GM and failing that, leave the group altogether.

    \subsection{Non-Player Characters}
    Non-Player Characters (NPCs) are characters in the story created by the GM or Players that act without Player input. Instead, the GM acts as the ``soul'' of every NPC and treats them as closely as a Player would treat their Character. GMs can oftentimes generate hundreds or even thousands of characters throughout long campaigns, so characters that are underdeveloped or single-purpose are acceptable as long as they are not used in a derogatory or offensive manner.

    \section{Players and their Characters}

    Players are the people in charge of Player Characters (PCs); their job is to be the ``soul'' driving their character in the direction most appropriate for them. Players are charged with the following responsibilities:

    \begin{itemize}
        \item Roleplaying their character.
        \item Being courteous to the Game Master and fellow Players.
        \item Knowing the rules within reason.
        \item Following the rules and decisions of the Game Master.
        \item Being honest about rolls and their character sheet.
    \end{itemize}

    \subsection{Player Characters}

    Player Characters (PCs) are the protagonists of any Arrata game. They exist to provide everyone with a point of view on the collective story being told and to allow the Player to interact with that story in accordance with how their character would behave. 
    
    Their PC is the primary responsibility of the Player, and thus if a conflict arises regarding your PC, it is your duty to respect the PC and fight on their behalf.

\end{document}