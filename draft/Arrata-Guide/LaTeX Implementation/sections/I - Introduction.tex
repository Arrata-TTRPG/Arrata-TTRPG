\documentclass[../main.tex]{subfiles}

\graphicspath{{\subfix{../images/}}}

\begin{document}

    This part contains all of the core mechanics of Arrata; detailing Roleplaying, Characters, their components, dice rolling, and how Characters change.

    \section{What is Arrata?}

    \thispagestyle{plain}

    Arrata is a roleplaying system inspired by the works of more traditional roleplaying games, with an emphasis on universality. The purpose of this system is to allow you to write stories with as much, or as little, nuance as you want.

    Arrata comes with no setting; I believe it is better to create something of your own and flesh it out as you go along. Whether that's characters, the world, or the context in which those interact, you will be able to do things far more suited to your understanding than I ever could.
    
    Doing things is often far more valuable and teachable than observing. Part of that is that {\em you will fail}. You will also succeed in creating your vision, although rarely in the manner you may be expecting. The great thing about fictional worlds is that they have no physical consequences, they cannot and will not hurt you, so you are free to try the things you want to, and to fail in awesome ways.

    In using this system, a fictional world is constructed by a {\em Game Master} (GM). This world is populated with characters and given things like factions, populations, and conflict, things that make it alive and interactable. {\em players} take control of {\em player characters} (PCs) who are built with {\em Quirks} that define who they are as people, and {\em Stats} that define what they are as beings that interact with the world. Together, the players and GM create a story with the PCs as the protagonists, their actions being influenced by their Quirks and the outcomes determined by their Stats.

    The word Arrata is a misspelling of errata, the plural of erratum; a list of mistakes in a written document. The word was chosen as it embodies the spirit of Arrata: \textbf{Change Through Purpose.} By failing; making mistakes and blunders, you will develop as a person and become better than before. I hope in some way this system will allow you to explore these ideas, and perhaps even learn how you can change yourself.

    \section{Game Masters}

    Game Masters (GMs) are a critical part of any roleplaying system. Their job is to:

    \begin{itemize}
        \item Understand the rules as thoroughly as possible.
        \item Roleplay Non-Player Characters (NPCs).
        \item Be courteous and fair to their Players.
        \item Provide a story and setting.
        \item Describe:
        \begin{itemize}
            \item The outcomes of rolls.
            \item The environment.
            \item NPCs and their actions.
            \item Reactions and consequences.
        \end{itemize}
    \end{itemize}
    
    The GM is the world engine, describing and defining what the world is: how it looks, smells, tastes, and sounds, and how it interacts with the Players' Characters and their actions.
    
    As a GM, you have the most responsibility; orchestrating sessions and campaigns, managing NPCs, handling disputes, etc. Your Players are counting on you to prepare and improvise as well as you can and if you can't do those things, I suggest being a Player.
    Game Masters are to be afforded extra rights over the Players. They will have to make rulings and decisions for the Players, and should act as a mediator; thus these rulings are to be respected and treated as the new rule of law unless otherwise changed by the GM.

    However, it is important not to overstep your authority as the GM. Punishing Players unfairly or making nonsensical rulings are unacceptable. If you find yourself under a GM making such decisions, the best course of action is typically to discuss the issue with the other Players and GM and failing that, leave the group altogether.

    \section{Players and their Characters}

    Players are the people in charge of player characters (PCs); their job is to be the ``soul'' driving their character in the direction most appropriate for them. Players are charged with the following responsibilities:

    \begin{itemize}
        \item Roleplaying their character.
        \item Being courteous to the Game Master and fellow players.
        \item Knowing the rules within reason.
        \item Following the rules and decisions of the Game Master.
        \item Being honest about rolls and their character sheet.
    \end{itemize}

    PCs are the protagonists of any Arrata game. They exist to provide a player with a point of view on the collective story being told and to allow that player to interact with that story following how their character would behave. Their PC is the primary responsibility of the Player, and thus if a conflict arises regarding your PC, it is your duty to respect the PC and fight on their behalf.

    \section{Non-Player Characters}
    Non-Player Characters (NPCs) are characters in the story created by the GM or players that act without player input. Instead, the GM acts as the ``soul'' of every NPC and treats them as closely as how a player would treat their character. 
    
    GMs can generate hundreds or even thousands of NPCs throughout long campaigns, so NPCs that are underdeveloped or single-purpose are acceptable as long as they are not used in a derogatory or offensive manner. On the other hand, situations may arise where an NPC is removed from the story when they were planned to have a greater role, in which case the GM shouldn't attempt to rewrite history and the flow of the story, they should accommodate and adapt the story to fit the new reality.

\end{document}