\documentclass[../main.tex]{subfiles}

\usepackage{blindtext}

\graphicspath{{\subfix{../images/}}}

\begin{document}

    \section{What is Arrata?}

    The word Arrata is a misspelling of errata, the plural of erratum; a list of mistakes in a written document.
    
    Entomology aside, Arrata is a Tabletop-Roleplaying Game (TTRPG) system with a focus on meaningful characters and their development. To emphasize, Arrata is a *system*, designed with tools to allow it to be used in many different settings. There are some tools for playing in more fantastical settings provided, but Arrata is certainly not limited to them.
    

    \section{What is Roleplaying?}

    Roleplaying is the act of putting yourself in the shoes of someone else. It's a form of improv where your job is to emulate and represent a character. There are many parts to roleplaying, such as the accent the character has, the ways they interact with different people, how they solve problems, etc.
    
    Anyone roleplaying a character should know as much about that character as they can if they hope to provide a reliable and accurate representation of them, and Arrata provides many tools that allow you to do this, along with helping you develop that character as they experience new things.

    \section{Game Masters (GMs)}

    Game Masters (GMs) are a critical part of any roleplaying system. Their job is to:

    \begin{itemize}
        \item Understand the system as thoroughly as possible.
        \item Roleplay Non-Player Characters (NPCs, also called Non-Playable Characters).
        \item Be courteous and fair to their Players.
        \item Provide a story and setting.
        \item Describe:
        \begin{itemize}
            \item the outcomes of rolls,
            \item the environment,
            \item reactions and consequences,
            \item everything that isn't already known and is wanted to be known.
        \end{itemize}
    \end{itemize}
    
    The GM's job is to be the world-engine, describing and defining what the world is, how it looks, and how it interacts with the players and their actions.
    
    As a GM, you have the most responsibility as you must know and/or improv everything the players and their characters will come to discover in your world. You also have the most amount of power, it's important to keep things fair and reasonable. It's also your job to act responsibly as a mediator if any disagreement or argument happens related to the game. The GM's rulings are absolute, so be sure to write them down if you know you'll come across them again.
    
    GMs typically orchestrate Campaigns and Sessions, which is an added layer of responsibility. To compensate this, they are afforded the right to remove Players they think are a poor fit or are being a nusance to the group.

    \section{Players and their Characters (PCs)}

    Players are people whose job it is to roleplay their character, and their characters are what the story being told follows.

    That may seem simple, but being a player has a lot different demands and expectations, including but not limited to:

    \begin{enumerate}
        \item Roleplaying their character.
        \item Being courteous to the Game Master and fellow players.
        \item Knowing the rules within reason.
        \item Following the rules and decisions of the Game Master.
        \item Being honest about rolls and their character sheet.
    \end{enumerate}

    Being a bad player is more than enough reason for a Game Master to remove them from their game - as a player you are owed nothing but the fate your rolls provide you.

    That doesn't mean that rolling bad should get you kicked; being an asshole will. Be reasonable, be understanding, and communicate with your GM if you have questions or concerns.

    On top of being a Player, you are assigned a Player Character (PC), which you will often times have the ability to create yourself. Your PC is the way in which you interact with the world, so do try to keep good care of them.

\end{document}